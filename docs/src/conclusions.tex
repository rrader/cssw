\section{Выводы}
    При виконанні курсового проекту розроблений програмний продукт для моделювання MPP систем у різних конфігураціях.

    Досліджені алгоритми сусіднього призначення із моделюванням та без моделювання. Ці алгоритми досліджені з використанням 3х алгоритмів формування черг.

    Показано, що алгоритм з моделюванням (6) дає найкращі результати, при використанні черги 3 від 0.1 до 0.5 показника зв'язності та при використанні черги 4 від 0.6 до 0.9.

    Суміщений алгоритм 6-3 та 6-4 на двох проміжках досліджений на різних системах: з топологією тор та решітка, з 1, 2 та 3 лінками на процесорах та із різними маштабами заданих задач, $1:1$, $1:2$, $1:3$.

    Показано, що на маштабі $1:3$ із 2 лінками на процесорах у топології тор алгоритм дає найбільш високий результат.