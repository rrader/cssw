\section{Опис алгоритмів планування}

Алгоритми планування складаються з двох етапів: перший етап складається з формування
черги завдань на основі графа завдань, а другий - у призначенні завдання з черги
на один з вільних процесорів.

\subsection{Алгоритми формування черг}
    Розглядаються три алгоритма формування черг.

    Алгоритм 3:
    У порядку спадання критичного по часу шляхів до кінця графа задачі.

    Алгоритм 4:
    У порядку спадання критичного по кількості вершин шляхів до кінця графа задачі, а при рівних значеннях – в порядку спадання зв’язності вершин.

    Алгоритм 16:
    У порядку зростання критичного по часу шляхів вершин від початку графа задачі.

    Розглянемо як рахуються такі показники як критичний шлях по часу до кінця, по
    кількості вершин до кінця, по часу від початку.

    \subsubsection{Критичний шлях по часу до кінця графа}
    Критичний шлях до кінця графа завдання - максимальна сума вагів вершин
    від якоїсь вершини вниз до кінця графа. Для обчислення можна використовувати пошук в глибину та вибрати
    найбільш важкий за сумою вагів вершин шлях. При цьому для зменшення кількості повторних обчислень необхідно
    зберігати шляхи для вже пройдених вершин. Таким критичний шлях для кожної вершини буде обчислений лиш один раз.

    \subsubsection{Критичний шлях по кількості вершин до кінця графа}
    Аналогічно до попереднього за винятком того, що замість суми вагів підраховується тільки кількість вершин.

    \subsubsection{Критичний шлях по часу від початку графа}
    Обернений показник до першого, замість пошуку шляху до кінця графу, необхідно знайти шлях від початку графа. При цьому
    поточна вершина не рахується.

    \subsubsection{Зв'язність вершини}
    Зв'язність вершини - кількість вхідних та вихідних дуг.

\subsection{Алгоритми призначення}
    В курсовому проекті розглядаються два алгоритми призначення, сусіднього призначення з пересилками із попередженням з аналітичним визначенням процесора, та з моделюванням для визначення початкового часу виконання задачі.

    \subsubsection{Алгоритм сусіднього призначення з пересилками із попередженням}
    5. Алгоритм «сусіднього» призначення із пересилками «з попередженням». У даному випадку використовується, як і в аналогічному алгоритмі для SMP систем, комунікаційна модель, коли дані передаються асинхронно відразу після їх формування. 

    Процесор обирається з вільних на даний момент процесорів по мінімальній сумі всіх пересилань на процесор для даної задачі (з урахуванням топології системи, тобто транзитних пересилок). Дані пересилаються з попередженням.

    \subsubsection{Алгоритм сусіднього призначення з моделюванням}
    6. Алгоритм «сусіднього» призначення з використанням моделювання для визначення початкового часу виконання обчислювальних робіт.

    Робиться перебір всіх вільних процесорів і спроба реального призначення на нього в даний момент. Процесор вибирається за мінімальним стартовому часу.
    Комунікаційна модель використовується така сама, як і в попередньому алгоритмі.

\subsection{Генерация випадкового графу задачі}
    Для генерування графу задачі  задаються наступні його параметри:
    \begin{itemize}
    \item мінімальна вага вершини графу;
    \item максимальна вага вершини;
    \item кількість вершин графу;
    \item зв’язність графу  (співвідношення часу виконання до часу пересилок);
    \item мінімальна вага дуг графа задачі (необов’язково);
    \item максимальна вага дуг графа задачі (необов’язково).
    \end{itemize}

    Зв'язність - параметр графу, який розраховується за формулою:
    
    \[ C = \frac{\sum_{i=1}^N{w_i}}{\sum_{i=1}^N{w_i} + \sum_{j=1}^M{e_j}}\]

    Таким чином, при зв'язності 1, ми маємо абсолютно незв'язних граф - що не містить
    жодної зв'язку, а зі зменшенням С - сумарна вага зв'язків зростатиме. це
    може досягатися або збільшенням їх кількості, або збільшенням їх ваги.

Алгоритм генерації випадкового графу задачі

\begin{enumerate}
    \item Генеруємо N вершин з випадковими вагами в заданих межах.
    \item Рахуємо суму вагів, за формулою кореляции рахуємо середню вагу дуги яку ми повинні отримати.
    \item По заданому відсотку кількості дуг та середному значенню, рахуємо їх кількість і генеруємо їх.
    \item Ваги дуг задаємо випадковим чином по нормальному розподілу, де mu = (очікувана сума) / (кількість дуг), а sigma = mu / 4, при цьому не дозволяємо вазі виходити за задані межі. При цьому, якщо середнє значення виходить за межі, то межі не враховуються.
    \item Якщо потрібна кореляція досягнута, але не всі ребра додані, зупиняємо додавання ребер.
    \item В останню дугу вагою записуємо необхідну різницю, щоб отримати потрібний коефіцієнт кореляції - без урахування меж.
\end{enumerate}
