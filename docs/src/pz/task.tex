\section*{Вступ}

Паралельна обробка інформації дає хороші перспективи для суттєвого збільшення продуктивності обчислювальних задач. Наразі створення найбільш ефективної паралельної системи є одним з найголовніших критеріїв конкурування виробників комп'ютерних систем. SMP системи обмежені в зростанні кількості процесорів, тому що різко зростає кількість конфліктів доступу до загальних ресурсів, як системна шина та пам'ять.

MPP системи не мають цього обмеження, тому кількість обчислювачів може зростати майже необмежено. Тим не менш, такі системи мають свої недоліки: відсутність загальної пам'яті потребує пересилань даних між процесорами. При великій кількості процесорів, повна зв'язність зазвичай не може бути забеспечена, за економічних причин і для забеспечення найбільшого коефіцієнту використання зв'язків. Тому необхідне планування пересилань між процесорами з транзитними пересилками.

Вочевидь це можна зробити різними способами, тому використання найбільш ефективного алгоритму планування обчислень та пересилок для певної задачі є важливим завданням для максимізації використання наданих ресурсів.
