\documentclass[ukrainian,utf8,nostitching,14pt]{eskdtext}

\usepackage[T2A]{fontenc}
\usepackage[utf8]{inputenc}
\usepackage{cyrtimes}
\usepackage{mathtext}
\usepackage{tabularx}
\usepackage{listings}
\usepackage{setspace}
\usepackage{graphicx}
\usepackage{indentfirst}
\usepackage{amsmath}
\usepackage{booktabs}
\usepackage{rotating}
\usepackage{float}
\ESKDsectStyle{section}{\Large\bfseries}
% \ESKDsectSkip{section}{10pt}{5pt}
\renewcommand{\baselinestretch}{1.5} % Задаём единичный межстрочный интервал

\usepackage{chngcntr}
\counterwithin{figure}{section}
\counterwithin{table}{section}

\ESKDsignature{ІАЛЦ 467449.002 ТЗ}
\ESKDauthor{Уваров Н.В.}
\ESKDchecker{Русанова О.В.}
\ESKDdocName{Технічне завдання}
\ESKDgroup{НТУУ КПИ \\ група ІО-41м}

\renewcommand{\labelenumi}{\arabic{enumi}. }

\begin{document}
\lstset{
basicstyle=\footnotesize,
numberstyle=\tiny,
tabsize=4,
breaklines=true,
title=\lstname
}

\section{Область застосування}
Розроблюваний програмний продукт призначений для дослідження алгоритмів планування обчислювальних робіт на різноманітних обчислювальних системах.

\section{Підстава для розробки}
Підставою для розробки є індивідуальне завдання на курсовий проект, видане кафедрою Обчислювальної техніки ФІОТ НТУУ “КПІ” п’ятому курсу спеціальності “Комп’ютерні системи та мережі” по дисципліні “Програмне забезпечення комп’ютерних систем”.

\section{Призначення розробки}
Даний програмний продукт призначений для демонстрації набутих навичок та закріплення отриманих знань по дисципліні “Програмне забезпечення комп’ютерних систем”.

\section{Вимоги до програми}
\subsection{Вимоги до функціональних характеристик}
Даний програмний продукт повинен забезпечувати:
графічне введення та редагування графу обчислювальної роботи та системи;
перевірку коректності введення вхідних даних;
генерування випадкового графу обчислювальної роботи згідно із заданими характеристиками;
планування обчислювальної роботи згідно із обраним алгоритмом та характеристиками обчислювальної системи;
виведення результатів планування обчислювальної роботи у вигляді діаграм Ґанта;
зрозумілий та зручний графічний інтерфейс користувача.
\subsection{Вимоги до надійності}
Програма повинна забезпечувати коректну роботу в середовищі Windows XP/Vista/7.
\subsection{Умови експлуатації}
        Всі експлуатаційні вимоги співпадають з вимогами експлуатації ПЕОМ IBM PC і сумісних з нею ПК.
\subsection{Вимоги до складу і параметрів технічних засобів}
Необхідна наявність IBM PC — сумісного ПК. Необхідний дисковий простір — 1 Мб.
\subsection{Вимоги до інформаційної та програмної сумісності}
Програмний продукт повинен функціонувати під управлінням 32-х та 64-х розрядних операційних систем сімейства Windows.
\subsection{Спеціальні вимоги}
Спеціальні вимоги до характеристик розробки не висуваються.


\section{Етапи та стадії розробки}

\begin{enumerate}
\item Узгодження технічного завдання.
\item Розробка програмного забезпечення.
\item Захист курсового проекту.
\end{enumerate}



\section{Перелік документації}
Опис альбому.
Технічне завдання.
Пояснювальна записка.



\end{document}
